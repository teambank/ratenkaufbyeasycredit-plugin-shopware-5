%% Generated by Sphinx.
\def\sphinxdocclass{report}
\documentclass[a4paper,10pt,openany,oneside,ngerman]{sphinxmanual}
\ifdefined\pdfpxdimen
   \let\sphinxpxdimen\pdfpxdimen\else\newdimen\sphinxpxdimen
\fi \sphinxpxdimen=.75bp\relax

\PassOptionsToPackage{warn}{textcomp}

\catcode`^^^^00a0\active\protected\def^^^^00a0{\leavevmode\nobreak\ }
\usepackage{cmap}
\usepackage{fontspec}
\usepackage{amsmath,amssymb,amstext}
\usepackage[ngerman]{babel}

\setmainfont{DejaVu Serif}
\setsansfont{DejaVu Sans}
\setmonofont{DejaVu Sans Mono}

\usepackage[Sonny]{fncychap}
\usepackage{sphinx}

\fvset{fontsize=auto}
\usepackage{geometry}

% Include hyperref last.
\usepackage{hyperref}
% Fix anchor placement for figures with captions.
\usepackage{hypcap}% it must be loaded after hyperref.
% Set up styles of URL: it should be placed after hyperref.
\urlstyle{same}
\addto\captionsngerman{\renewcommand{\contentsname}{Inhaltsverzeichnis}}

\addto\captionsngerman{\renewcommand{\figurename}{Abb.}}
\addto\captionsngerman{\renewcommand{\tablename}{Tab.}}
\addto\captionsngerman{\renewcommand{\literalblockname}{Quellcode}}

\addto\captionsngerman{\renewcommand{\literalblockcontinuedname}{Fortsetzung der vorherigen Seite}}
\addto\captionsngerman{\renewcommand{\literalblockcontinuesname}{Fortsetzung auf der nächsten Seite}}

\addto\extrasngerman{\def\pageautorefname{Seite}}

\setcounter{tocdepth}{1}


\usepackage[titles]{tocloft}
\cftsetpnumwidth {1.25cm}\cftsetrmarg{1.5cm}
\setlength{\cftchapnumwidth}{0.75cm}
\setlength{\cftsecindent}{\cftchapnumwidth}
\setlength{\cftsecnumwidth}{1.25cm}


\title{ratenkauf by easyCredit für Shopware}
\date{28.03.2019}
\release{1.3.0}
\author{Teambank AG}
\newcommand{\sphinxlogo}{\sphinxincludegraphics{50_ratenkauf_Logo_1240x824_easyCredit.png}\par}
\renewcommand{\releasename}{Release}
\makeindex

\begin{document}
\ifnum\catcode`\"=\active\shorthandoff{"}\fi
\maketitle
\sphinxtableofcontents
\phantomsection\label{\detokenize{index::doc}}



\chapter{Voraussetzungen}
\label{\detokenize{requirements:ratenkauf-by-easycredit-fur-shopware-5-x}}\label{\detokenize{requirements::doc}}\label{\detokenize{requirements:voraussetzungen}}
Zu Nutzung von \sphinxstylestrong{ratenkauf by easyCredit für Shopware} benötigen Sie:
\begin{itemize}
\item {} 
eine funktionierende \sphinxstylestrong{Shopware 5.x} Installation

\item {} 
valide API-Zugangsdaten zu \sphinxstyleemphasis{ratenkauf by easyCredit}

\end{itemize}


\section{Versionskompatibilität}
\label{\detokenize{requirements:versionskompatibilitat}}
Das Plugin hat neben Shopware 5.x keine weiteren Abhängigkeiten. Es wurde für die folgenden Versionen mit dem Responsive Template getestet:
\begin{itemize}
\item {} 
5.0.x

\item {} 
5.1.x

\item {} 
5.2.x

\item {} 
5.3.x

\item {} 
5.4.x

\item {} 
5.5.x

\end{itemize}

Das Emotion Template und Shopware Versionen \textless{} 5.0 werden von der aktuellen Version nicht unterstützt.
Verwenden Sie für solche Installationen bitte das ältere Release v1.2.5 für Shopware 4.


\section{Zugangsdaten für „ratenkauf by easyCredit“}
\label{\detokenize{requirements:zugangsdaten-fur-ratenkauf-by-easycredit}}
Um ratenkauf by easycredit nutzen zu können, benötigen Sie gültige API-Zugangsdaten.

Nehmen Sie einfach Kontakt mit uns auf unter:
\begin{quote}

\sphinxhref{mailto:sales.ratenkauf@easycredit.de}{sales.ratenkauf@easycredit.de}

+49(0)911/5390 2726
\end{quote}

Oder registrieren Sie sich direkt unter:
\begin{quote}

\sphinxurl{https://www.easycredit-ratenkauf.de/registrierung.htm}
\end{quote}

und wir setzen uns mit Ihnen in Verbindung.


\chapter{Installation}
\label{\detokenize{installation:installation}}\label{\detokenize{installation::doc}}
Die Extension für ratenkauf by easyCredit kann im Plugin-Manager entweder über den direkten Download aus dem \sphinxstyleemphasis{Shopware Community Store} oder über den Datei-Upload des bereitgestellten Archives über \sphinxstyleemphasis{Plugin Hochladen} installiert werden.
Alternativ ist auch die Installation über die Kommandozeile möglich.


\section{Shopware Community Store}
\label{\detokenize{installation:shopware-community-store}}
Zu Installation melden Sie sich im Backend Ihrer Shopware Installation an. Sie finden das Backend unter der folgenden URL: \sphinxcode{\sphinxupquote{http(s)://mein-shop.tld/backend}}.
Öffnen Sie im Backend nun den Plugin-Manager durch Klick im Menü auf \sphinxmenuselection{Einstellungen \(\rightarrow\) Plugin-Manager}.

\begin{sphinxadmonition}{note}{Bemerkung:}
Als Shortcut verwenden Sie die Tastenkombination \sphinxtitleref{Strg + Alt + P}.
\end{sphinxadmonition}

Sie finden das Plugin, indem Sie in Suche oben links \sphinxstylestrong{easyCredit} eingeben:

\noindent\sphinxincludegraphics{{installation-community_store}.png}

Das Plugin wird Ihnen nun zur Installation angezeigt. Installieren Sie das Plugin nun durch Klick auf \sphinxstylestrong{Installieren}.
Die Extension wird automatisch heruntergeladen und installiert.

Fahren Sie anschließend mit der {\hyperref[\detokenize{configuration:configuration}]{\sphinxcrossref{\DUrole{std,std-ref}{Konfiguration}}}} fort.


\section{manueller Datei-Upload}
\label{\detokenize{installation:manueller-datei-upload}}
Gehen Sie analog zur Installation aus dem Shopware Community Store vor. Im Plugin-Manager wählen Sie im linken Menü \sphinxmenuselection{Verwaltung \(\rightarrow\) Installiert}. Klicken Sie nun auf den Button \sphinxstyleemphasis{Plugin hochladen}. Wählen Sie den lokalen Pfad aus, unter dem sich das ZIP-Archiv des Shopware Plugins befindet und klicken Sie anschließend auf \sphinxstyleemphasis{Plugin hochladen}.

Fahren Sie anschließend mit der {\hyperref[\detokenize{configuration:configuration}]{\sphinxcrossref{\DUrole{std,std-ref}{Konfiguration}}}} fort.

\noindent\sphinxincludegraphics{{installation-file_upload}.png}


\section{Kommandozeile}
\label{\detokenize{installation:kommandozeile}}
Um das Plugin über die Kommandozeile zu installieren, führen Sie die folgenden Befehle aus:

\fvset{hllines={, ,}}%
\begin{sphinxVerbatim}[commandchars=\\\{\}]
\PYG{g+gp}{\PYGZdl{}} wget https://www.easycredit\PYGZhy{}ratenkauf.de/download/easycredit\PYGZhy{}shopware\PYGZhy{}1.3.0.zip
\PYG{g+gp}{\PYGZdl{}} cp easycredit\PYGZhy{}shopware\PYGZhy{}1.3.0.zip /sw\PYGZhy{}base\PYGZhy{}dir/engine/Shopware/Plugins/
\PYG{g+gp}{\PYGZdl{}} \PYG{n+nb}{cd} /sw\PYGZhy{}base\PYGZhy{}dir/engine/Shopware/Plugins/
\PYG{g+gp}{\PYGZdl{}} unzip easycredit\PYGZhy{}shopware\PYGZhy{}1.3.0.zip
\PYG{g+gp}{\PYGZdl{}} rm easycredit\PYGZhy{}shopware\PYGZhy{}1.3.0.zip
\end{sphinxVerbatim}

Um sicher zu gehen, überprüfen Sie, ob das folgende Verzeichnis existiert: \sphinxcode{\sphinxupquote{engine/Shopware/Plugins/Netzkollektiv/EasyCredit}}. Im Anschluss installieren und aktivieren Sie das Plugin mit den folgenden Befehlen:

\fvset{hllines={, ,}}%
\begin{sphinxVerbatim}[commandchars=\\\{\}]
\PYG{g+gp}{\PYGZdl{}} \PYG{n+nb}{cd} /sw\PYGZhy{}base\PYGZhy{}dir
\PYG{g+gp}{\PYGZdl{}} ./bin/console sw:plugin:refresh
\PYG{g+gp}{\PYGZdl{}} ./bin/console sw:plugin:install NetzkollektivEasyCredit
\PYG{g+gp}{\PYGZdl{}} ./bin/console sw:plugin:activate NetzkollektivEasyCredit
\end{sphinxVerbatim}

Fahren Sie anschließend mit der {\hyperref[\detokenize{configuration:configuration}]{\sphinxcrossref{\DUrole{std,std-ref}{Konfiguration}}}} fort.

Sollten Ihnen die Zugangsdaten bereits vorliegen, können Sie diese gleich bei der Installation mit den folgenden Befehlen setzen:

\fvset{hllines={, ,}}%
\begin{sphinxVerbatim}[commandchars=\\\{\}]
\PYG{g+gp}{\PYGZdl{}} ./bin/console sw:plugin:config:set NetzkollektivEasyCredit easycreditApiKey 1.de.1234.4321
\PYG{g+gp}{\PYGZdl{}} ./bin/console sw:plugin:config:set NetzkollektivEasyCredit easycreditApiToken abc\PYGZhy{}def\PYGZhy{}ghi
\end{sphinxVerbatim}


\chapter{Konfiguration}
\label{\detokenize{configuration:konfiguration}}\label{\detokenize{configuration:configuration}}\label{\detokenize{configuration::doc}}
Nachdem Sie die Installation erfolgreich abgeschlossen haben, konfigurieren Sie das Plugin. Damit das Plugin als Zahlungsmethode angezeigt wird aktivieren Sie ratenkauf by easyCredit als Zahlungsmethode für den deutschen Store.


\section{Konfigurations Menü öffnen}
\label{\detokenize{configuration:konfigurations-menu-offnen}}
Zur Konfiguration öffnen Sie im Backend erneut den Plugin-Manager. In der Liste der installierten Plugins sollte nun \sphinxstylestrong{ratenkauf by easyCredit} enthalten sein.
In dieser Zeile klicken Sie das Stifte Icon, um die Plugin Konfiguration zu öffnen.

\noindent\sphinxincludegraphics[scale=0.5]{{config-open}.png}


\section{Plugin konfigurieren}
\label{\detokenize{configuration:plugin-konfigurieren}}
Die Konfigurationsmöglichkeiten sind im Folgenden gezeigt und in der Tabelle im einzelnen beschrieben. Als Mindestkonfiguration geben Sie hier Ihren API-Key und Ihren API-Token an.
Sollten Sie Einstellungen vorgenommen habem, so speichern Sie die Einstellungen mit einem Klick auf \sphinxstylestrong{Speichern}.

\noindent\sphinxincludegraphics{{config-dialog}.png}


\begin{savenotes}\sphinxattablestart
\centering
\begin{tabulary}{\linewidth}[t]{|p{85pt}|J|}
\hline

Option
&
Erklärung
\\
\hline
Zeige Modellrechner-Widget neben Produktpreis
&
Aktivieren Sie diese Option wenn Sie auf Produkt-Detail-Seiten ein monatliches Raten Angebot anzeigen möchten. Bitte beachten Sie das ein monatlicher Ratenpreis nur angezeigt wird wenn der Preis des Produkts sich in der festgelegten Preisspanne für Ratenkäufe befindet.
\\
\hline
Bestellungsstatus
&
Ermöglicht es Ihnen den Status festzulegen den Bestellungen die mit \sphinxstylestrong{ratenkauf by easyCredit} bezahlt wurden, nach dem Eingang im System aufweisen.
\\
\hline
Zahlungsstatus
&
Ermöglicht es Ihnen den Zahlungsstatus festzulegen den Bestallungen die mit \sphinxstylestrong{ratenkauf by easyCredit} bezahlt wurden, nach dem Eingang im System aufweisen.
\\
\hline
Debug Logging
&
Erlaubt Ihnen festzulegen ob der Inhalt aller easyCredit API-Zugriffe werden soll. Fehlermitteillungen werden immer gespeichert.
\\
\hline
API Key
&
Der API-Key wird Ihnen von der Teambank AG zur Verfügung gestellt.
\\
\hline
API Token
&
Der nicht öffentliche API Token wird Ihnen von der Teambank AG zur Verfügung gestellt und sollte nicht mit Dritten geteilt werden.
\\
\hline
easyCredit Zugangsdaten überprüfen
&
Ein Klick auf diesen Button überprüft die Kombination von API-Key und -Token auf Gültigkeit. Bitte vergessen Sie nicht nach einem erfolgreichen Test noch auf Speichern zu klicken.
\\
\hline
\end{tabulary}
\par
\sphinxattableend\end{savenotes}

\begin{sphinxadmonition}{note}{Bemerkung:}
Nach einem erfgolreichen Test der API-Zugangsdaten, vergessen Sie bitte nicht auf \sphinxstylestrong{Speichern} zu klicken.
\end{sphinxadmonition}


\section{Zahlungsart Einstellungen}
\label{\detokenize{configuration:zahlungsart-einstellungen}}
Um die Zahlungsart \sphinxstylestrong{ratenkauf by easyCredit} im Frontend anzuzeigen, muss die Zahlungsart aktiviert sein, und dem Land \sphinxstyleemphasis{Deutschland} zugewiesen werden. Navigieren Sie hierzu zu den Zahlungsart Einstellungen: \sphinxmenuselection{System -\textgreater{} Konfiguration -\textgreater{} Zahlungsarten -\textgreater{} ratenkauf by easyCredit}
Im ersten Reiter \sphinxstylestrong{Generell} stellen Sie sicher, dass \sphinxstylestrong{ratenkauf by easyCredit} aktiviert ist.

\noindent\sphinxincludegraphics{{config-payment-active}.png}

\clearpage

Aktivieren Sie als letzten Schritt nun im Reiter \sphinxstylestrong{Länder-Auswahl} das Land \sphinxstylestrong{Deutschland}.

\noindent\sphinxincludegraphics{{config-payment-country}.png}


\chapter{Frequently Asked Questions}
\label{\detokenize{faq::doc}}\label{\detokenize{faq:frequently-asked-questions}}

\section{Nach Auswahl der Zahlungsart im Frontend, wird die Zustimmungserklärung nicht angezeigt}
\label{\detokenize{faq:nach-auswahl-der-zahlungsart-im-frontend-wird-die-zustimmungserklarung-nicht-angezeigt}}
Wird die Zahlungsart in der Zahlartenauswahl durch Klick auf den Radio-Button ausgewählt, erscheint bei korrekter Funktionalität eine Zustimmungserklärung, in der der Kunde seine Zustimmung zur Übermittlung seiner p
ersönlichen Daten an die Server der Teambank AG zustimmt. Diese Übermittlung ist notwendig, um den Kunden an das Zahlungsterminal weiterzuleiten, um ihm seinen Ratenwunsch zu berechnen.
\begin{description}
\item[{Wird diese Zustimmungserklärung nicht angezeigt, ist der Fehler meist im verwendeten Template zu finden. Im Template \sphinxcode{\sphinxupquote{themes/Frontend/MeinTheme/frontend/checkout/shipping\_payment.tpl}} muss an erster Stelle stehen:}] \leavevmode
\sphinxcode{\sphinxupquote{\{extends file="parent:frontend/checkout/shipping\_payment.tpl"\}}}. Nur so erbt das Template vom für die confirm-Seite vorgesehenen Template.

\end{description}

\begin{sphinxadmonition}{note}{Bemerkung:}
Beginnt das Template mit \sphinxcode{\sphinxupquote{\{extends file="frontend/index/index.tpl"\}}} so erbt das Template von allgemeinen Standardtemplate. In diesem Fall wird die Zustimmungserklärung nicht angezeigt.
\end{sphinxadmonition}


\section{Nach Aktivieren der Debug Logging Option in den Plugin-Einstellungen, wird kein Debug Output geloggt}
\label{\detokenize{faq:nach-aktivieren-der-debug-logging-option-in-den-plugin-einstellungen-wird-kein-debug-output-geloggt}}
Das Log Level im Production Modus von Shopware ist standardmäßig so eingestellt, dass nur kritische Fehler in den Logfiles erscheinen.
Durch folgende Anpassung des Log Levels in der \sphinxtitleref{config.php} werden auch weniger kritische Nachricht geloggt:

\fvset{hllines={, ,}}%
\begin{sphinxVerbatim}[commandchars=\\\{\}]
\PYG{x}{\PYGZsq{}logger\PYGZsq{} =\PYGZgt{} [}
\PYG{x}{   \PYGZsq{}level\PYGZsq{} =\PYGZgt{} 100}
\PYG{x}{]}
\end{sphinxVerbatim}


\section{Nach Auswahl der Zahlungsart im Frontend, wird der Tilgungsplan auf der Bestellbestätigungsseite nicht angezeigt}
\label{\detokenize{faq:nach-auswahl-der-zahlungsart-im-frontend-wird-der-tilgungsplan-auf-der-bestellbestatigungsseite-nicht-angezeigt}}
\begin{sphinxadmonition}{note}{Bemerkung:}
Dieser Hinweis gilt nur für ältere Versionen des Plugins (\textless{}= v1.5.x)
\end{sphinxadmonition}

Wurde die Zahlungsart vom Kunden augewählt und der Ratenwunsch über das Ratenkaufterminal bestätigt, wird dem Kunden der Tilgungsplan auf der Bestätigungsseite angezeigt.

Wird der Tilgungsplan nicht angezeigt, ist der Fehler meist im verwendeten Template zu finden. Im Template \sphinxtitleref{themes/Frontend/MeinTheme/frontend/checkout/confirm.tpl} muss an erster Stelle stehen: \sphinxcode{\sphinxupquote{\{extends file="par
ent:frontend/checkout/confirm.tpl"\}}}. Nur so erbt das Template vom für die confirm-Seite vorgesehenen Template.

\begin{sphinxadmonition}{note}{Bemerkung:}
Beginnt das Template mit \sphinxcode{\sphinxupquote{\{extends file="frontend/index/index.tpl"\}}} so erbt das Template von allgemeinen Standardtemplate. In diesem Fall wird der Tilgungsplan nicht angezeigt.
\end{sphinxadmonition}


\chapter{Changelog}
\label{\detokenize{changelog::doc}}\label{\detokenize{changelog:changelog}}

\section{v1.5.3}
\label{\detokenize{changelog:v1-5-3}}\begin{itemize}
\item {} 
die Zinsen werden nun nach einem Abbruch der Bestellung / Wechsel der Zahlungsart zuverlässig entfernt (siehe \#3594)

\item {} 
Bestellstatus und Zahlungsstatus Dropdown zeigen ihre Werte nun zuverlässig an (siehe \#3592)

\item {} 
der „Modus“ (Artikeltyp) der Zinsen wird nach Bestellung angepasst, um ein korrektes Steuerhandling in Rechnung zu erreichen

\item {} 
die Zustimmungserklärung wird nun pro Store gecacht (Multi-Store Kompatibilität)

\end{itemize}


\section{v1.5.2}
\label{\detokenize{changelog:v1-5-2}}\begin{itemize}
\item {} 
Möglichkeit der Änderung der Adresse bei nicht akzeptierten Adressen oder Adresskombinationen über konditional eingeblendete Lightbox (\#3526)

\item {} 
Angabe einer abweichenden Lieferadresse im Bestätigungsschritt ist nicht mehr möglich bei Zahlart ratenkauf by easyCredit

\item {} 
die statische Zustimmungserklärung wird einen Tag im Shop des Händlers gecacht, bevor ein neuer Request an die API erfolgt (Performance)

\end{itemize}


\section{v1.5.1}
\label{\detokenize{changelog:v1-5-1}}\begin{itemize}
\item {} 
Möglichkeit hinzugefügt, Ratenkaufzinsen im Backend automatisch aus Bestellungen und in Rechnungen zu entfernen

\item {} 
Fehlermeldungen werden nicht mehr als Snippets ausgegeben

\end{itemize}


\section{v1.5.0}
\label{\detokenize{changelog:v1-5-0}}\begin{itemize}
\item {} 
Anpassungen zur Kompatibilität mit Shopware 5.5 RC 1

\item {} 
das Widget-Plugin wurde durch eine neue Version ersetzt (Entfernung von Bootstrap zur Reduzierung des Konfliktpotentials)

\item {} 
die Fehlermeldung bei Ändern der Lieferadresse im Backend wird nun zuverlässig angezeigt

\item {} 
bei Anpassung der Standard-Zahlungsmethode im Kundenaccount wird die Zustimmungserklärung nicht mehr angezeigt

\item {} 
obsolete Funktionen wurden entfernt

\end{itemize}


\section{v1.4.9}
\label{\detokenize{changelog:v1-4-9}}\begin{itemize}
\item {} 
das Widget kann nun, ohne Leeren des Caches, zuverlässig deaktiviert/aktiviert werden

\end{itemize}


\section{v1.4.8}
\label{\detokenize{changelog:v1-4-8}}\begin{itemize}
\item {} 
Verbesserung der Kompatibilität mit aktuellen und zukünftigen Versionen von Shopware

\item {} 
Verbessertes Handling von Zahlartenabschlägen in Verbindung mit dem ratenkauf by easyCredit

\item {} 
Angleichung des Wordings zum easyCredit Händlerinterface

\end{itemize}


\section{v1.4.7}
\label{\detokenize{changelog:v1-4-7}}\begin{itemize}
\item {} 
Anpassung von Links wegen Website Relaunch

\end{itemize}


\section{v1.4.6}
\label{\detokenize{changelog:v1-4-6}}\begin{itemize}
\item {} 
Verbesserung der Kompatibilität mit aktuellen und zukünftigen Versionen von Shopware

\end{itemize}


\section{v1.4.4}
\label{\detokenize{changelog:v1-4-4}}\begin{itemize}
\item {} 
behebt ein Problem, dass das Speichern von ratenkauf by easyCredit Bestellungen im Backend verhindert hat

\item {} 
zuverlässigere Anzeige des Ratenkauf-Widgets durch Verwendung eines anderen Events

\end{itemize}


\section{v1.4.3}
\label{\detokenize{changelog:v1-4-3}}\begin{itemize}
\item {} 
behebt fehlerhaftes Verhalten in bestimmten Umgebungen (Checkout zeigt weisse Seite, \#3418)

\item {} 
optimierte Darstellung der Zahlungsart (Payment Selection \& Confirm-Seite)

\item {} 
Anpassung zur Verwendung mit Custom Products Plugin (Produkte ohne Preis werden nicht an API gesendet)

\item {} 
Code Cleanup: entfernt Verweise auf altes Emotion Template

\item {} 
Widget wird auch bei deaktiviertem asynchronem JS-Loading angezeigt

\item {} 
Performance-Optimierung Widget

\end{itemize}


\section{v1.4.1}
\label{\detokenize{changelog:v1-4-1}}\begin{itemize}
\item {} 
\#3408: Upgrade Anzeige in Shopware Marketplace ist für dieses Modul korrekt

\item {} 
\#3408: JS Fehler, wenn Modul als Letztes in Zahlungsarten-Auswahl

\item {} 
doppelte Anzeige des Widgets in manchen Umgebungen

\item {} 
Upgrade der API-Library

\item {} 
behebt ein Fehlverhalten, wenn API Warning zurückliefert

\end{itemize}


\section{v1.3.0}
\label{\detokenize{changelog:v1-3-0}}\begin{itemize}
\item {} 
Shopware 5.3.x Kompatibilität

\item {} 
kein Support mehr für Shopware 4.x

\end{itemize}


\section{v1.2}
\label{\detokenize{changelog:v1-2}}\begin{itemize}
\item {} 
Shopware 5.2.x Kompatibilität

\item {} 
Rechtliche API-Übertragungsnachricht wird vom easyCredit Server dynamisch abgerufen

\item {} 
easyCredit API v4

\end{itemize}


\section{v1.1}
\label{\detokenize{changelog:v1-1}}\begin{itemize}
\item {} 
Kompatibilitättests

\end{itemize}


\section{v1.0}
\label{\detokenize{changelog:v1-0}}\begin{itemize}
\item {} 
erstes öffentliches Release

\end{itemize}



\renewcommand{\indexname}{Stichwortverzeichnis}
\footnotesize\raggedright\printindex
\end{document}