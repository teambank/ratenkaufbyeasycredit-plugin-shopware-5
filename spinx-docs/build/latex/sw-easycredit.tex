%% Generated by Sphinx.
\def\sphinxdocclass{report}
\documentclass[a4paper,10pt,openany,oneside,ngerman]{sphinxmanual}
\ifdefined\pdfpxdimen
   \let\sphinxpxdimen\pdfpxdimen\else\newdimen\sphinxpxdimen
\fi \sphinxpxdimen=.75bp\relax

\PassOptionsToPackage{warn}{textcomp}

\catcode`^^^^00a0\active\protected\def^^^^00a0{\leavevmode\nobreak\ }
\usepackage{cmap}
\usepackage{fontspec}
\usepackage{amsmath,amssymb,amstext}
\usepackage[ngerman]{babel}

\setmainfont{DejaVu Serif}
\setsansfont{DejaVu Sans}
\setmonofont{DejaVu Sans Mono}

\usepackage[Sonny]{fncychap}
\usepackage{sphinx}

\fvset{fontsize=auto}
\usepackage{geometry}

% Include hyperref last.
\usepackage{hyperref}
% Fix anchor placement for figures with captions.
\usepackage{hypcap}% it must be loaded after hyperref.
% Set up styles of URL: it should be placed after hyperref.
\urlstyle{same}
\addto\captionsngerman{\renewcommand{\contentsname}{Inhaltsverzeichnis}}

\addto\captionsngerman{\renewcommand{\figurename}{Abb.}}
\addto\captionsngerman{\renewcommand{\tablename}{Tab.}}
\addto\captionsngerman{\renewcommand{\literalblockname}{Quellcode}}

\addto\captionsngerman{\renewcommand{\literalblockcontinuedname}{Fortsetzung der vorherigen Seite}}
\addto\captionsngerman{\renewcommand{\literalblockcontinuesname}{Fortsetzung auf der nächsten Seite}}

\addto\extrasngerman{\def\pageautorefname{Seite}}

\setcounter{tocdepth}{1}


\usepackage[titles]{tocloft}
\cftsetpnumwidth {1.25cm}\cftsetrmarg{1.5cm}
\setlength{\cftchapnumwidth}{0.75cm}
\setlength{\cftsecindent}{\cftchapnumwidth}
\setlength{\cftsecnumwidth}{1.25cm}


\title{ratenkauf by easyCredit für Shopware}
\date{04.04.2018}
\release{1.3.0}
\author{Teambank AG}
\newcommand{\sphinxlogo}{\sphinxincludegraphics{50_ratenkauf_Logo_1240x824_easyCredit.png}\par}
\renewcommand{\releasename}{Release}
\makeindex

\begin{document}
\ifnum\catcode`\"=\active\shorthandoff{"}\fi
\maketitle
\sphinxtableofcontents
\phantomsection\label{\detokenize{index::doc}}



\chapter{Voraussetzungen}
\label{\detokenize{requirements:ratenkauf-by-easycredit-fur-shopware-5-x}}\label{\detokenize{requirements::doc}}\label{\detokenize{requirements:voraussetzungen}}
Zu Nutzung von \sphinxstylestrong{ratenkauf by easyCredit für Shopware} benötigen Sie:
\begin{itemize}
\item {} 
eine funktionierende \sphinxstylestrong{Shopware 5.x} Installation

\item {} 
valide API-Zugangsdaten zu \sphinxstyleemphasis{ratenkauf by easyCredit}

\end{itemize}


\section{Versionskompatibilität}
\label{\detokenize{requirements:versionskompatibilitat}}
Das Plugin hat neben Shopware 5.x keine weiteren Abhängigkeiten. Es wurde für die folgenden Versionen mit dem Responsive Template getestet:
\begin{itemize}
\item {} 
5.0.x

\item {} 
5.1.x

\item {} 
5.2.x

\item {} 
5.3.x

\item {} 
5.4.x

\end{itemize}

Das Emotion Template und Shopware Versionen \textless{} 5.0 werden von der aktuellen Version nicht unterstützt.
Verwenden Sie für solche Installationen bitte das ältere Release v1.2.5 für Shopware 4.


\section{Zugangsdaten für „ratenkauf by easyCredit“}
\label{\detokenize{requirements:zugangsdaten-fur-ratenkauf-by-easycredit}}
Um ratenkauf by easycredit nutzen zu können, benötigen Sie gültige API-Zugangsdaten.

Nehmen Sie einfach Kontakt mit uns auf unter:
\begin{quote}

\sphinxhref{mailto:sales.ratenkauf@easycredit.de}{sales.ratenkauf@easycredit.de}

+49(0)911/5390 2726
\end{quote}

Oder registrieren Sie sich direkt unter:
\begin{quote}

\sphinxurl{https://www.easycredit-ratenkauf.de/registrierung.htm}
\end{quote}

und wir setzen uns mit Ihnen in Verbindung.


\chapter{Installation}
\label{\detokenize{installation:installation}}\label{\detokenize{installation::doc}}
Das Plugin kann im Plugin-Manager entweder über den direkten Download aus dem \sphinxstylestrong{Community Store} oder über den \sphinxstylestrong{Datei-Upload „Plugin Hochladen“} der gepackten Extension installiert werden.
Alternativ ist auch die Installation über die Kommandozeile oder FTP möglich.

Zur Installation melden Sie sich im Backend Ihrer Shopware Installation an. Sie finden das Backend unter:
\begin{quote}

\sphinxcode{\sphinxupquote{http(s)://mein-shop.de/backend}}
\end{quote}

Öffnen Sie im Backend nun den Plugin-Manager. Klicken Sie dazu auf den folgenden Menüpunkt:
\begin{quote}

\sphinxmenuselection{Einstellungen \(\rightarrow\) Plugin-Manager}
\end{quote}


\section{Installation über Shopware Community Store}
\label{\detokenize{installation:installation-uber-shopware-community-store}}
\noindent\sphinxincludegraphics{{installation-community_store}.png}
\begin{itemize}
\item {} 
Zur Installation über den Shopware Community Store suchen Sie das Plugin im Community Store.

\item {} 
Klicken Sie nach Auswahl des Plugins den Button \sphinxstylestrong{Installieren}.

\item {} 
Die Extension wird nun heruntergeladen und installiert.

\item {} 
Fahren Sie anschließend mit der \sphinxstyleemphasis{Konfiguration} fort.

\end{itemize}


\section{Installation über Datei-Upload}
\label{\detokenize{installation:installation-uber-datei-upload}}
\noindent\sphinxincludegraphics{{installation-file_upload}.png}
\begin{itemize}
\item {} 
Zur Installation über den Plugin-Upload wählen Sie im Plugin-Manager den Menu-Punkt \sphinxmenuselection{Verwaltung \(\rightarrow\) Installiert}

\item {} 
Klicken Sie dort den Button \sphinxstylestrong{Plugin hochladen}.

\item {} 
Wählen Sie nun das Zip-Archiv, dass Sie von unserer Website heruntergeladen haben aus

\item {} 
und klicken Sie auf \sphinxstyleemphasis{Plugin hochladen}.

\end{itemize}

Beispiel:
\begin{quote}

\sphinxcode{\sphinxupquote{C:\textbackslash{}easycredit-shopware-1.3.0.zip}}
\end{quote}
\begin{itemize}
\item {} 
Fahren Sie anschließend mit der \sphinxstyleemphasis{Konfiguration} fort.

\end{itemize}


\section{Installation über Kommandozeile}
\label{\detokenize{installation:installation-uber-kommandozeile}}
Entpacken Sie das Zip-Archive

\fvset{hllines={, ,}}%
\begin{sphinxVerbatim}[commandchars=\\\{\}]
\PYG{n+nv}{\PYGZdl{} }cp easycredit\PYGZhy{}shopware\PYGZhy{}x.x.zip /shopware/engine/Shopware/Plugins/
\PYG{n+nv}{\PYGZdl{} }\PYG{n+nb}{cd} /shopware/engine/Shopware/Plugins/
\PYG{n+nv}{\PYGZdl{} }unzip easycredit\PYGZhy{}shopware\PYGZhy{}x.x.zip
\PYG{n+nv}{\PYGZdl{} }rm easycredit\PYGZhy{}shopware\PYGZhy{}x.x.zip
\end{sphinxVerbatim}

Anschließend überprüfen Sie ob das folgende Verzeichnis existiert:
\begin{quote}

\sphinxcode{\sphinxupquote{engine/Shopware/Plugins/Netzkollektiv/EasyCredit}}
\end{quote}

Führen Sie nun folgende Befehle aus:

\fvset{hllines={, ,}}%
\begin{sphinxVerbatim}[commandchars=\\\{\}]
\PYG{n+nv}{\PYGZdl{} }/shopware/bin/console sw:plugin:refresh
\PYG{n+nv}{\PYGZdl{} }/shopware/bin/console sw:plugin:install NetzkollektivEasyCredit
\PYG{n+nv}{\PYGZdl{} }/shopware/bin/console sw:plugin:activate NetzkollektivEasyCredit
\end{sphinxVerbatim}

Fahren Sie anschließend mit der Konfiguration fort.


\chapter{Konfiguration}
\label{\detokenize{configuration:konfiguration}}\label{\detokenize{configuration::doc}}
Die Konfiguration teilt sich auf in die Plugin-Konfiguration und die Zahlarten-Konfiguration.


\section{Plugin konfigurieren}
\label{\detokenize{configuration:plugin-konfigurieren}}
\noindent\sphinxincludegraphics[scale=0.5]{{config-open}.png}
\begin{itemize}
\item {} 
Öffnen Sie im Backend den Plugin-Manager. \sphinxmenuselection{Einstellungen -\textgreater{} Plugin-Manager}

\item {} 
In der Liste der installierten Plugins sollte nun \sphinxstylestrong{ratenkauf by easyCredit} erscheinen.

\item {} 
Öffnen Sie die Plugin-Konfiguration über das Stifte Icon (ganz rechts).

\end{itemize}

Für eine Grundkonfiguration tragen Sie zumindest die Zugangsdaten (API Key \& API Token) ein.
Testen Sie die Zugangsdaten mit Klick auf \sphinxstylestrong{Zugangsdaten testen}.
Nach dem erfolgreichen Test klicken Sie auf \sphinxstylestrong{Speichern}

\noindent\sphinxincludegraphics{{config-dialog}.png}


\begin{savenotes}\sphinxattablestart
\centering
\begin{tabulary}{\linewidth}[t]{|p{85pt}|J|}
\hline

Option
&
Erklärung
\\
\hline
Zeige Modellrechner-Widget neben Produktpreis
&
Aktivieren Sie diese Option wenn Sie auf Produkt-Detail-Seiten ein monatliches Raten Angebot anzeigen möchten. Bitte beachten Sie das ein monatlicher Ratenpreis nur angezeigt wird wenn der Preis des Produkts sich in der festgelegten Preisspanne für Ratenkäufe befindet.
\\
\hline
Bestellungsstatus
&
Ermöglicht es Ihnen den Status festzulegen den Bestellungen die mit \sphinxstylestrong{ratenkauf by easyCredit} bezahlt wurden, nach dem Eingang im System aufweisen.
\\
\hline
Zahlungsstatus
&
Ermöglicht es Ihnen den Zahlungsstatus festzulegen den Bestallungen die mit \sphinxstylestrong{ratenkauf by easyCredit} bezahlt wurden, nach dem Eingang im System aufweisen.
\\
\hline
Debug Logging
&
Erlaubt Ihnen festzulegen ob der Inhalt aller easyCredit API-Zugriffe werden soll. Fehlermitteillungen werden immer gespeichert.
\\
\hline
API Key
&
Der API-Key wird Ihnen von der Teambank AG zur Verfügung gestellt.
\\
\hline
API Token
&
Der nicht öffentliche API Token wird Ihnen von der Teambank AG zur Verfügung gestellt und sollte nicht mit Dritten geteilt werden.
\\
\hline
easyCredit Zugangsdaten überprüfen
&
Ein Klick auf diesen Button überprüft die Kombination von API-Key und -Token auf Gültigkeit. Bitte vergessen Sie nicht nach einem erfolgreichen Test noch auf Speichern zu klicken.
\\
\hline
\end{tabulary}
\par
\sphinxattableend\end{savenotes}


\section{Zahlungsart Einstellungen}
\label{\detokenize{configuration:zahlungsart-einstellungen}}
Im Shopware Backend öffnen Sie den Zahlungsarten Eintrag für \sphinxstylestrong{ratenkauf by easyCredit}
\begin{quote}

\sphinxmenuselection{System -\textgreater{} Konfiguration -\textgreater{} Zahlungsarten -\textgreater{} ratenkauf by easyCredit}
\end{quote}

Im ersten Reiter ‚\sphinxstyleemphasis{Generell}‘ stellen Sie sicher, dass \sphinxstylestrong{ratenkauf by easyCredit} aktiviert ist.

\noindent\sphinxincludegraphics{{config-payment-active}.png}

\clearpage

Aktivieren Sie weiterhin im Reiter \sphinxstyleemphasis{Länder-Auswahl} das Land Deutschland.

\noindent\sphinxincludegraphics{{config-payment-country}.png}


\chapter{Changelog}
\label{\detokenize{changelog::doc}}\label{\detokenize{changelog:changelog}}

\section{v1.3.0}
\label{\detokenize{changelog:v1-3-0}}\begin{itemize}
\item {} 
Shopware 5.3.x Kompatibilität

\item {} 
kein Support mehr für Shopware 4.x

\end{itemize}


\section{v1.2}
\label{\detokenize{changelog:v1-2}}\begin{itemize}
\item {} 
Shopware 5.2.x Kompatibilität

\item {} 
Rechtliche API-Übertragungsnachricht wird vom easyCredit Server dynamisch abgerufen

\item {} 
easyCredit API v4

\end{itemize}


\section{v1.1}
\label{\detokenize{changelog:v1-1}}\begin{itemize}
\item {} 
Kompatibilitättests

\end{itemize}


\section{v1.0}
\label{\detokenize{changelog:v1-0}}\begin{itemize}
\item {} 
erstes öffentliches Release

\end{itemize}



\renewcommand{\indexname}{Stichwortverzeichnis}
\footnotesize\raggedright\printindex
\end{document}